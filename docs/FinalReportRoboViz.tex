%%%%%%%%%%%%%%%%%%%%%%%%%%%%%%%%%%%%%%%%%%%%%%%%%%%%%%%%%%%%%%%%%
%%%  Capstone Project Template that tries to save a few trees %%%
%%%  Edwin Blake 22 Aug 2013                                  %%%
%%%		1 Aug 2014 (revised)                                  %%%
%%%             10 Aug 2015                                   %%%
%%%  see also                                                 %%%
%%% http://ravirao.wordpress.com/2005/11/19/latex-tips-to-meet-publication-page-limits/
%%%%%%%%%%%%%%%%%%%%%%%%%%%%%%%%%%%%%%%%%%%%%%%%%%%%%%%%%%%%%%%%%

\documentclass[11pt,a4paper]{article}
\usepackage{times}
% Allows better control over headers and footers
\usepackage{fancyhdr}
% set the margins using the geometry package (which is much the easiest way of
% doing this).
\usepackage[margin=2.5cm]{geometry}
% Pictures (means you have to produce pdf output via pdflatex)
\usepackage[pdftex]{graphicx}
% Set a default path for the image files
\graphicspath{{./images/}}
% Clickable hyperlinks
\usepackage{hyperref}
% Try to reduce the white space latex loves so much
\usepackage[small,compact]{titlesec}
% Reduce space around section heads and add a full stop after the number
\titlelabel{\thetitle. \quad}
% Use fancy headers
\pagestyle{fancy}
% Change name of Abstract to nothing and loose some of the excessive white
% space
\renewcommand{\abstractname}{\vskip -5mm}

\begin{document}
\title{Final Report: RoboViz Capstone Project} \date{}
\author{Boyd Kane\\KNXBOY001\\KNXBOY001@myuct.ac.za
\and Imaad Ghoor\\GHRIMA002\\GHRIMA002@myuct.ac.za
\and Jesse Sarembock\\SRMJES001\\SRMJES001@myuct.ac.za}

%  Set the headers via fancyhdr package
% Short title for running head
\lhead{RoboViz Final Report}
\chead{}
\lfoot{}
% add page number as centre footer.
\cfoot{\thepage}
\rfoot{}
% Don't want horizontal line under header
\renewcommand{\headrulewidth}{0.0pt}

\maketitle
% First page is plain style headings and footers (ie just the page number as
% footer).
\thispagestyle{plain}

\begin{abstract}
% First you should have an executive summary (or abstract) just a single
% paragraph saying what the results of the project are (at most 200
% words).
    This Final Report describes the 2021 RoboViz Capstone project completed by
    third year computer science students. This project expanded on the open
    source RoboGen software to add the ability to simulate multiple robots in a
    swarm, among other extensions.
\end{abstract}

% We expect a report of about 3500-4000 words, written single spaced, with a font
% size of at least 11 pts.  Use at least a 2.5 cm margin on all sides of the
% pages.
%
% No blank lines between paragraphs except to get figures and their captions to
% position properly.
%
% Depending on how many diagrams you use (more is better) the report will be
% between 7 and 10 pages long. Your appendices (e.g., user manual, test results,
% which are needed) are not included in these limits.
%
% You must had-in an Adobe Acrobat file for your report (i.e., pdf
% file).
% You should begin your write-up with an overview and then drill down
% into the details of what you produced. Your report should cover the
% following sections (Sections \ref{s:introduction} --
% \ref{s:conclusion}).
%
% Some notes about code formatting
% - Each method should start with a brief description of its
%   function.
% - Use indentation to display the structure within a method.
% - Comments should be used extensively. They are best used to
%   describe logical blocks of code rather than individual
%   statements. Line-by-line comments have the drawbacks of not
%   providing any overview and of decreasing readability.
% - Meaningful identifiers should be chosen.
% - Output should be pleasingly formatted and easy to read.
%
% You do, of course, have the option to call in any of your
% favourite packages for setting maths, graphics, computer listings,
% etc.

\section{Introduction}
\label{s:introduction}
% Your introduction provides the context for the project and should
% contain the statement of the scope of the project (which may have
% changed since you first wrote it). Someone reading your introduction
% must have clear idea of what the system is intended for. If you think
% there is something special about the kind of problem you tackled that
% your reader needs to know up front then this is where you say it.
%
% If you need any survey of other work (you probably don't) then put it
% towards the end of the introduction and give suitable references. A
% case where this is needed is if your project builds on someone else's
% project or some published algorithm.
%
% Discuss your approach to solving the problem. Please give a short
% overview of the software engineering methods you used (e.g.,
% traditional analysis followed by design and implementation -- typically
% the case if you did an evolutionary prototype, or a more agile
% approach where you had a cyclical development process).
This is where the introduction will go

\section{Requirements Captured}
% The next section deals with the analysis of your system. Cover the
% functional, non-functional and usability requirements. This is where
% you present your use case narratives and diagrams.

% Discuss the major analysis artefacts that you produced. We will expect
% you to produce at least one overall description of the architecture
% used in your system as a diagram, either here or below (see Section
% \ref{s:design-overview}). You may also want to include an analysis
% class hierarchy diagram.
This is where the requirements captured will go.

\section{Design Overview}
\label{s:design-overview}
% The next section is an overview of your design. The system design has
% to be justified in terms of the expected behaviour of the final
% product.
%
% If you produced a design class diagram put it here.
% \begin{figure}[h!]
%   \center{\includegraphics[scale=0.8]{architecture.png}}
%   \caption{An architecture diagram. Caption to go below figure}
%   \label{fig:architecture}
% \end{figure}
%
% You must present the overall architecture of the system together with
% an architecture diagram. You may choose what kind of diagram best
% suits your project but we would expect a layered architecture diagram
% (see Figure \ref{fig:architecture}) unless there is a good reason for
% some other kind of diagram. It need not be a formal UML diagram as
% long as it conveys all the necessary information clearly.
%
% You should then (in subsections) cover the algorithms and the data
% organisation used and why they were considered the best.
This is where the design overview should go

\section{Implementation}
% Now we get to the details.
% - Describe your data structures and be sure to illustrate them with a
%     diagram.
% - If your user interface was a key feature describe how that was
%     implemented.
% - Discuss the function of the most significant methods in each class.
%     This may well require flowcharts, or sequence diagrams, in some cases.
% - Any special relationship between the classes (e.g. friends) and why
%     they exist.
% - A description of any special programming techniques or libraries
%     used.
This is where the implementation details should go.

\section{Program Validation and Verification}
\label{s:progr-valid-verif}
% Tell us how you tested the system and why you believe it works.
% Describe the Quality Management Plan for your project, that is,
% software testing plan. The plan should indicate the types of testing
% that was performed and detail how they were done. This must include
% the reasons on why the chosen testing protocol was considered
% effective.
%
% Create a table that summarizes the testing plan (see Table
% \ref{tab:test-plan}).
%
% \begin{table}[h!]
%     \centering
%     \caption{Summary Testing Plan. A table caption goes above the table.}
%
%     \begin{tabular}[t]{|p{8cm}|p{7cm}|} \hline
%         \textbf{Process} & \textbf{Technique} \\ \hline 1. Class
%         Testing: test methods and state behaviour of classes & Random,
%         Partition and White-Box Tests \\ \hline 2. Integration Testing:
%         test the
%         interaction of sets of classes & Random and Behavioural Testing \\
%         \hline 3. Validation Testing: test whether customer requirements
%         are satisfied & Use-case based black box and Acceptance tests \\
%         \hline 4. System Testing: test the behaviour of the system as part
%         of a larger environment & Recovery, security, stress and
%         performance tests \\ \hline
%     \end{tabular}
%
%     \label{tab:test-plan}
% \end{table}
% Describe all the steps taken to validate the correctness of the
% program.
%
% If you had user tests then say what you did and what the results
% were. Describe why these test data were chosen (what test conditions
% the data was testing).  Table \ref{tab:tests} provides an example of
% the sorts of results we are looking for. The full detail of the test
% runs should be appended to the report.
This is where we prove the program is valid and that we've tested it.

% \begin{table}[h!]
%     \centering
%     \caption{A table of tests. A table caption goes above the table.}
%
%     \begin{tabular}[t]{|p{5.5cm}|p{3cm}|p{3cm}|p{3cm}|} \hline \textbf{Data Set
%         and reason for its choice} & \multicolumn{3}{c|}{\textbf{Test Cases}}\\
%         \cline{2-4} & \emph{Normal Functioning} & \emph{Extreme boundary cases} &
%         \emph{Invalid Data (program should not crash)} \\ \hline Preliminary test
%         (see Appendix 3) & Passed & n/a & Fell over \\\hline &&&\\ \hline
%                          &&&\\ \hline
%     \end{tabular}
%
%     \label{tab:tests}
% \end{table}
% Follow your table of results with a discussions of them highlighting
% how useful and usable your system is for its intended purpose.

\section{Conclusion}
\label{s:conclusion}
% Your report must have a clear conclusion where you revisit the aims
% set out in the beginning and discuss how well you met them. Did you
% achieve the objective of creating a well-structured, modular, and
% robust system?  Please summarize the design features and test results
% that show this.
Here we must summarize everything, and provide a conclusion to the project's
initial aims and goals
\appendix
\section{User Manual}
\label{s:user-manual}
This user manual still needs to be included.
% Your system must have a user manual. Append this to your report (make
% it Appendix A) or bind it separately if it is big. If your system is
% interactive and has a good user interface with context dependent help
% then this can be just a cheat sheet. Discuss the level at which your
% user manual is to be pitched with your client. If your system is to be
% extended then you might want to include a technical API manual.

\end{document}
