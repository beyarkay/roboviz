% Your system must have a user manual. Append this to your report (make
% it Appendix A) or bind it separately if it is big. If your system is
% interactive and has a good user interface with context dependent help
% then this can be just a cheat sheet. Discuss the level at which your
% user manual is to be pitched with your client. If your system is to be
% extended then you might want to include a technical API manual.

% TODO Add the user manual as appendix A
% TODO discuss the level at which the user manual is to be pitched
% TODO If the system is to be extended, add an API manual




\begin{table}[h!]
    \centering
    \caption{Overview of User Manual}
    \begin{tabular}[t]{|p{8cm}|p{7cm}|} \hline
        \textbf{Section} & \textbf{Description} \\
        \hline 1. Introduction
                         & \\
        \hline 2. Definitions
                            & Description of the tools involved when running the program. \\
        \hline 3. Installation 
                            & Instructions on installing the software. \\ 
        \hline 4.  Running RoboViz
                            & Instructions on running the code. \\
        \hline 5.  Using the Visualiser
                            & Description of controls used to navigate the visualiser. \\
        \hline
    \end{tabular}
    \label{tab:T}
\end{table}

\subsection{Introduction}
\label{s:introduction}

The purpose of the modified version of RoboViz is to enable the visualisation of multiple robots in the task environment, as opposed to
a single robot. The number of robots that can be held in a swarm depends on how powerful the system that is running it.



\subsection{Definitions}
\label{s:definitions}

Robot File - A file that defines a robot and its composition.

Configuration File - File that defines the behaviour of the robots when being visualised.

Robot FileViewer or Visualier - The window where the user will be able to view the robots performing.



\subsection{Installation}
\label{s:installation}

1. Open the RoboViz folder
2. Navigate to the build folder, if there is no build folder, open the terminal in that location and type \texttt{cd build}
3. %Open terminal, if you haven't already, and type \textt{ cmake -DCMAKE_BUILD_TYPE=Release -G"Unix Makefiles" ../src/ }
4. Type \texttt{make -j3}
5. Once that is done, you should see the following message \texttt{Linking CXX executable robogen-server}


 


\subsection{Running RoboViz}
\label{s:runningroboviz}

To start viewing the robot in the task environment, you'll need to enter the commands displayed in the pictures below.

\begin{figure}[htpb]
    \centering
    \includegraphics[width=0.8\textwidth]{run}
    \caption{commands in terminal }
    \label{fig:commands-in-terminal}
\end{figure}

Once entered, the follwing screen should be displayed.

\begin{figure}[htpb]
    \centering
    \includegraphics[width=0.8\textwidth]{visual1}
    \caption{Robots in Visualiser}
    \label{fig:robots-in-visualiser}
\end{figure}

